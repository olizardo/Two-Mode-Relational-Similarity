%% 
%% Copyright 2019-2020 Elsevier Ltd
%% 
%% This file is part of the 'CAS Bundle'.
%% --------------------------------------
%% 
%% It may be distributed under the conditions of the LaTeX Project Public
%% License, either version 1.2 of this license or (at your option) any
%% later version.  The latest version of this license is in
%%    http://www.latex-project.org/lppl.txt
%% and version 1.2 or later is part of all distributions of LaTeX
%% version 1999/12/01 or later.
%% 
%% The list of all files belonging to the 'CAS Bundle' is
%% given in the file `manifest.txt'.
%% 
%% Template article for cas-dc documentclass for 
%% double column output.

%\documentclass[a4paper,fleqn,longmktitle]{cas-dc}
\documentclass[a4paper,fleqn]{cas-sc}

%\usepackage[numbers]{natbib}
%\usepackage[authoryear]{natbib}
\usepackage[authoryear,longnamesfirst]{natbib}
\usepackage{graphicx}
\usepackage{caption}
\usepackage{subcaption}

%%%Author definitions
\def\tsc#1{\csdef{#1}{\textsc{\lowercase{#1}}\xspace}}
\tsc{WGM}
\tsc{QE}
\tsc{EP}
\tsc{PMS}
\tsc{BEC}
\tsc{DE}
%%%

% Uncomment and use as if needed
%\newtheorem{theorem}{Theorem}
%\newtheorem{lemma}[theorem]{Lemma}
%\newdefinition{rmk}{Remark}
%\newproof{pf}{Proof}
%\newproof{pot}{Proof of Theorem \ref{thm}}

\begin{document}
\let\WriteBookmarks\relax
\def\floatpagepagefraction{1}
\def\textpagefraction{.001}

% Short title
\shorttitle{Two-Mode Relational Similarities}

% Short author
%\shortauthors{O. Lizardo}

% Main title of the paper
\title [mode = title]{Two-Mode Relational Similarities}                      
% Title footnote mark
% eg: \tnotemark[1]
%\tnotemark[1,2]

% Title footnote 1.
% eg: \tnotetext[1]{Title footnote text}
% \tnotetext[<tnote number>]{<tnote text>} 
%\tnotetext[1]{}

%\tnotetext[2]{.}


% First author
%
% Options: Use if required
% eg: \author[1,3]{Author Name}[type=editor,
%       style=chinese,
%       auid=000,
%       bioid=1,
%       prefix=Sir,
%       orcid=0000-0000-0000-0000,
%       facebook=<facebook id>,
%       twitter=<twitter id>,
%       linkedin=<linkedin id>,
%       gplus=<gplus id>]
\author[1]{Omar Lizardo}[orcid=0000-0001-7511-2910]

% Corresponding author indication
\cormark[1]

% Footnote of the first author
\fnmark[1]

% Email id of the first author
\ead{olizardo@soc.ucla.edu}

% URL of the first author
\ead[url]{http://olizardo.bol.ucla.edu/}

%  Credit authorship
\credit{Conceptualization of this study, Methodology, Software}

% Address/affiliation
\affiliation[1]{organization={University of California, Los Angeles},
    addressline={264 Haines Hall}, 
    city={Los Angeles},
    %citysep={}, % Uncomment if no comma needed between city and postcode
     postcode={90095}, 
     state={CA},
     country={USA}}

% Corresponding author text
\cortext[cor1]{Corresponding author}
%\cortext[cor2]{Principal corresponding author}

% Footnote text
\fntext[fn1]{This paper was written while the author was partially supported by National Science Foundation grant HNDS-R \#2214215.}
 % author as well.}
%\fntext[fn2]{Another author footnote, this is a very long footnote and
 % it should be a really long footnote. But this footnote is not yet
 % sufficiently long enough to make two lines of footnote text.}

% For a title note without a number/mark
%\nonumnote{This note has no numbers. In this work we demonstrate $a_b$
%  the formation Y\_1 of a new type of polariton on the interface
 % between a cuprous oxide slab and a polystyrene micro-sphere placed
 % on the slab.
 % }
% Here goes the abstract
\begin{abstract}
In a previous paper, Kovacs (2010) proposed a generalized relational similarity measure based on iterated correlations of entities in a network calibrated by their relational similarity to other entities. Here I show that, in the case of two-mode network data, Kovacs's approach can be simplified and generalized similarities calculated non-iteratively. The basic idea is to rely on initial similarities calculated from transforming the two-mode data into one-mode projections using the familiar duality approach due to Breiger (1974). I refer to this as two-mode relational similarities and show, using the Southern Women's data and data from Senate voting in the 112th U.S. Congress, that it yields results substantively indistinguishable from Kovacs's iterative strategy. 
\end{abstract}

% Use if graphical abstract is present
% \begin{graphicalabstract}
% \includegraphics{figs/grabs.pdf}
% \end{graphicalabstract}
% Keywords
% Each keyword is separated by \sep
\begin{keywords}
Generalized similarity \sep Duality \sep Two-mode networks \sep Correlation distance \sep Projection
\end{keywords}


\maketitle
\end{document}